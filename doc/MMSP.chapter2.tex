% MMSP.chapter2.tex

\chapter{Getting started with \MMSP}
The following sections provide the necessary information for new users to obtain and set up \MMSP.  Typically, this involves downloading a source code archive, unpacking it, and running a few tests.  Developers or those interested in maintaining an up-to-date version of the code should consider checking out a copy from the {\tt git} repository.

\section{Download}
The source code for \MMSP\ is available through a {\tt git} repository, hosted on \href{https://github.com/mesoscale/mmsp}{GitHub}.  This requires the \href{http://www.git-scm.com}{\tt git} version control software to be installed on the target machine, as well as a working internet connection.  From the command line, type
\begin{shadebox}
\begin{verbatim}
    git clone https://github.org/mesoscale/mmsp.git
\end{verbatim}
\end{shadebox}
If the checkout is successful, the latest version of \MMSP\ will be copied into a sub-directory named {\tt mmsp}. There is no need to perform the steps for installation described below, and the user should move on to setup.

Having a local copy of \MMSP\ makes it simple to keep your code up-to-date.  From the root folder, simply type
\begin{shadebox}
\begin{verbatim}
    git pull
\end{verbatim}
\end{shadebox}
to update a working copy with the latest version of the \MMSP\ source code.  See the {\tt git} documentation for more details on version control.

\section{Installation}
If {\tt git} is unavailable or not desired, the user should download the \MMSP\ source code archive by clicking the link to ``Download ZIP'' from  \href{https://github.com/mesoscale/mmsp}{https://github.com/mesoscale/mmsp}.  Installation should be as simple as unpacking the archive.  Users with administrator privileges may choose to install the \MMSP\ header files in a location that will be searched by their compiler's preprocessor, but we do not describe how to do this here.  The following paragraphs provide platform-specific instructions for a local installation.

\paragraph{Linux/Unix}
Local installation for Linux users should simply involve unpacking the archive.  As most Linux systems have means to unpack both tarballs and zip files, there is likely no reason to prefer either.  After downloading the archive file, move it to the directory where you want \MMSP\ to reside, making sure that you have read access to the file as well as write access to the directory.  Then issue a command to unpack the archive, e.g.
\begin{shadebox}
\begin{verbatim}
    unzip mmsp-master.zip && mv mmsp-master mmsp
\end{verbatim}
\end{shadebox}
This will unpack the contents of the archive into a folder named {\tt mmsp}.  Next, type
\begin{shadebox}
\begin{verbatim}
    ls mmsp
\end{verbatim}
\end{shadebox}
which should indicate that folders such as {\tt mmsp/doc}, {\tt mmsp/examples}, etc.~have been created.  If either command fails or the folder {\tt mmsp} is not created, check the {\tt tar} or {\tt unzip} documentation.

\paragraph{Mac OS}
Mac OS users will follow much of the same procedure as Linux users, so it is advisible to read the previous section on Linux installation.  For those uninitiated, or who have never had any previous reason to use it, the {\tt Terminal} application can be found under {\tt Applications/Utilities}.  Again, all steps described above for Linux installation should apply here as well.

\paragraph{Windows}
For those who insist on using Windows, it is still possible to use \MMSP.  The preferred way to do this is to use the \href{http://www.cygwin.com}{cygwin} environment.  To use {\tt cygwin} with \MMSP, it is necessary that appropriate packages have been installed.  The following packages are required to run \MMSP\ : 

\begin{itemize}
  \item \textbf{gcc-g++}
  \item \textbf{make}
  \item \textbf{zlib-devel}
  \item \textbf{libpng-devel}
  \item \textbf{openmpi}
  \item \textbf{libopenmpi-devel}
\end{itemize}

If {\tt cygwin} has been installed properly, \MMSP\ may be installed by following the steps described above for Linux installations.

It is also possible to compile \MMSP\ source code within a code development environment such as Visual Studio, however, \MMSP\ code is typically so simple that any code management beyond command line or makefile compilation is only a hinderance.

\section{Setup}
Once \MMSP\ has been installed, there are a few useful tests and utility programs that should be generated.  First, enter the directory {\tt MMSP/utility} and type
\begin{shadebox}
\begin{verbatim}
    make
\end{verbatim}
\end{shadebox}
This will produce a number of conversion programs.  These include
\begin{itemize}
  \item {\tt mmsp2vti}, which converts \MMSP\ grid data files into formats that can be read by visualization software such as \href{http://www.paraview.org}{\tt ParaView},
  \item {\tt mmsp2png}, which converts \MMSP\ grid data files Portable Network Graphics (PNG), and
  \item several utilities for translating between different types of \MMSP\ grid data.
\end{itemize}
Because the programs provided in this directory are used so often, \MMSP\ users may wish to add the {\tt MMSP/utility} directory to their command path.  This can be achieved by adding the following lines to their {\tt \$HOME/.bashrc} file,
\begin{shadebox}
\begin{verbatim}
    export PATH=$PATH:$HOME/mmsp/utility
\end{verbatim}
\end{shadebox}
for users of the {\tt bash} shell, or
\begin{shadebox}
\begin{verbatim}
    setenv PATH $PATH:$HOME/mmsp/utility
\end{verbatim}
\end{shadebox}
for users of the {\tt tcsh} shell.  If you cloned or extracted \MMSP\ into a path other than {\tt ~/mmsp}, e.g. {\tt ~/Downloads/mmsp}, specify the appropriate path in the environmental variable.

Next, enter the {\tt MMSP/test} directory and type
\begin{shadebox}
\begin{verbatim}
    make test
\end{verbatim}
\end{shadebox}
or, if {\tt make} is not installed, type
\begin{shadebox}
\begin{verbatim}
    g++ -I ../include test.cpp -o test
\end{verbatim}
\end{shadebox}
If this compiles with no problems, then you're in luck; issue the command
\begin{shadebox}
\begin{verbatim}
    ./test
\end{verbatim}
\end{shadebox}
which will generate a short message indicating successful operation.

If the test program fails to compile, it is most likely because either {\tt make} or {\tt g++} (the GNU {\tt c++} compiler) is not installed on the system or is not configured properly.  Of course, any other ISO-compliant {\tt c++} compiler may be used instead.  If there is a problem at this stage, users should check their configuration.

Those who plan to use \MMSP\ with the MPI (Message Passing Interface) libraries should also take this opportunity to test their MPI configuration.  To do this, type
\begin{shadebox}
\begin{verbatim}
    make parallel
\end{verbatim}
\end{shadebox}
which, if successful, produces a parallel version of the test program.
If {\tt make} is not installed, type
\begin{shadebox}
\begin{verbatim}
    mpic++ -I../include -include mpi.h  test.cpp -o parallel
\end{verbatim}
\end{shadebox}
Once the code is compiled, run the program using an appropriate command for your MPI distribution, e.g.  
\begin{shadebox}
\begin{verbatim}
    mpirun -np 4 parallel
\end{verbatim}
\end{shadebox}
Note that for successful compilation, the MPI distribution is expected to provide a compilation script named {\tt mpic++} and a header file named {\tt mpi.h}.  If the program fails to compile, it may be that the user's MPI distribution provides {\tt mpicxx}, {\tt mpiCC}, or the like instead of {\tt mpic++}, or that it provides {\tt mpicxx.h} or something similar instead of {\tt mpi.h}.  In this case, the user should edit the {\tt Makefile} accordingly.  Likewise, the appropriate command to run the compiled program may differ depending on the MPI distribution.  This may take the form of, e.g.\ {\tt mpiexec}, instead of {\tt mpirun}.  If the user has multiple versions of MPI installed (e.g., OpenMPI and MPICH2) side-by-side, runtime errors will occur if the program compiles against one library and executes against the other.  Managing such an environment is beyond the scope of this document, but the \MMSP\ developers may be able to assist over e-mail.

Finally, the full \MMSP\ codebase can be tested by compiling and running the example programs.  While this can be done one-by-one, it is more efficient to execute the {\tt build\_examples.sh} script by typing
\begin{shadebox}
\begin{verbatim}
    ./build_examples.sh
\end{verbatim}
\end{shadebox}
The script takes the following actions for each example:
\begin{enumerate}
  \item Compile for serial using {\tt g++}
  \item Compile for parallel using {\tt mpic++}
  \item Execute in parallel, taking 1000 timesteps
  \item Convert each of the stored checkpoint files to PNG for visual inspection
  \item Summarize compiler and runtime errors, if encountered.
\end{enumerate}
The script accepts command-line arguments, summarized in the first few lines of the script.  Open it in a text editor to see the options and recommended usage.  This same script is used on the \href{http://travis-ci.org}{Travis Continuous Integration} service to test the \MMSP\ source code and proposed changes for compiler and runtime problems.  If the script encounters errors on your machine, you can check whether the problem is local or systematic by visiting the \MMSP\ project page at \href{https://travis-ci.org/mesoscale/mmsp}{Travis-CI} or \href{https://github.com/mesoscale/mmsp}{GitHub}.

\section{Advanced Setup}
Users are encouraged to copy the \MMSP\ example implementations as starting points for new projects.  To protect the work from being over-written when \MMSP\ is updated, the user is advised to copy the example code into a directory outside the \MMSP\ installation path, then set appropriate environmental variables in the {\tt bash} profile and the {\tt Makefile}.  To do this, append {\tt \$HOME/.bashrc} with
\begin{shadebox}
\begin{verbatim}
    export MMSP_PATH=$HOME/mmsp
\end{verbatim}
\end{shadebox}
and, in the project {\tt Makefile}, replace statements similar to 
\begin{shadebox}
\begin{verbatim}
    incdir=../../../../include
\end{verbatim}
\end{shadebox}
with
\begin{shadebox}
\begin{verbatim}
    incdir=$(MMSP_PATH)
\end{verbatim}
\end{shadebox}
Finally, if customization of {\tt main()} is needed, copy the generic {\tt MMSP.main.hp} using
\begin{shadebox}
\begin{verbatim}
    cp $MMSP_PATH/include/MMSP.main.hpp main.cpp
\end{verbatim}
\end{shadebox}
then specify the new filename, {\tt main.cpp}, in the {\tt \#include} statement at the foot of the {\tt .cpp} file.  Build your custom project with one of
\begin{shadebox}
\begin{verbatim}
    make
    make parallel
\end{verbatim}
\end{shadebox}

\section{Support}
\MMSP\ is not commercial code and there are no guarantees or claims, stated or implied, pertaining to its fitness for any purpose.  \MMSP\ is intended soley for use in non-profit scientific research.  In spite of this, the \MMSP\ team is devoted to producing a quality product that addresses the needs of the scientific community.  Please do not hesitate to contact our development team with any questions or suggestions.  Contact information can be found on the \MMSP\ page at \href{https://github.com/mesoscale}{GitHub}.

