% MMSP.chapter6.tex

\chapter{\MMSP\ Examples}

\section{Introduction}
	The \MMSP\ source code includes a large number of examples implementing
	canonical problems in diffusion, phase transformation, and statistical
	mechanics. These are provided to demonstrate the best practices for
	writing efficient \MMSP\ code and to provide starting points for your
	research. These examples are documented as clearly as possible in the
	{\tt c++} files themselves, with references to academic research papers
	for further details.
	
	\MMSP\ provides a common back-end for solving partial differential equations
	on spatial domains. Writing the equation of motion is up to the user: \MMSP\ 
	itself is agnostic to your realm of inquiry or numerical techniques. Example
	folder may contain several related approaches: kinetic Monte Carlo vs. phase
	field, isotropic vs. anisotropic, vector vs. sparse data types.
	
	These scripts are intended to showcase \MMSP, not to test the physics \emph{per se}.
	You are encouraged to adapt an example for your research purposes, but
	check our work against the literature. For consistency, every example compiles
	against {\tt MMSP/include/MMSP.main.hpp}, which serves as a useful template for
	your own {\tt main()} function.
	

\section{Algorithms}
	The folder {\tt MMSP/algorithms} contains miscellaneous algorithms
	for intializing and analyzing simulation domains, especially for grain
	growth models. Some highlights include:
	\begin{itemize}
		\item[{\tt tessellate.hpp}] Algorithms to initialize an empty grid with
		a Poisson-Voronoi tessellation, a commonly used starting point for grain
		growth simulations.
		\item[{\tt grainsize.hpp}] Algorithms to compute grainsize.
		\item[{\tt topology}] Algorithms to extract discrete topological
		properties from diffuse interfaces.
	\end{itemize}
	
\section{Beginner's Diffusion}
	Solves the canonical diffusion-type equation,
	\[\frac{\partial C}{\partial t} = -D\nabla^2C.\]

\section{Coarsening}
	\subsection{Grain growth}
		\subsubsection{Phase field}
			Evolves co-existing sub-domains $\phi_i\in\vec{\phi}$ according to non-conserved
			Allen-Cahn dynamics with mobility $L$,
			\begin{align*}
				\frac{\delta\mathcal{F}}{\delta\phi_i} &= -\epsilon^2\nabla^2\phi_i + \phi_i^3 - \phi_i^2 + \phi_i\sum\limits_{j\neq i}\phi_j^2 \
			    	                                    = -\epsilon^2\nabla^2\phi_i - \phi_i^2 + \phi_i\sum\limits_{j=1}^N\phi_j^2\\
				\frac{\partial\phi_i}{\partial t} &= -L\frac{\delta\mathcal{F}}{\delta\phi_i}.
			\end{align*}
		
	\subsection{Ostwald ripening}
		Evolves coupled equations for non-conserved (phase $\phi$) and conserved
		(composition $c$) quantities.
		Due to surface curvature, larger domains consume smaller ones to reduce the
		total system energy.

	\subsection{Zener pinning}

\section{Phase Transitions}
	\subsection{Allen-Cahn}
	\subsection{Cahn-Hilliard}
		Applies Cahn-Hilliard dynamics to the spinodal defined by 
		\[f(c) = \frac{C^4}{4} - \frac{C^2}{2},\]
		producing a fourth-order PDE:
		\[\frac{\partial C}{\partial t} = D\nabla^2\left(\frac{\partial f}{\partial C} - \epsilon^2\nabla^2 C\right).\]

		\subsubsection{Explicit}
			Discretizes the Cahn-Hilliard equation explicitly, coupling two second-order equations
			in the old value $(C^n)$ and new value $(C^{n+1})$ to recover the fourth-order dynamics:
			\begin{align*}
				\mu &= \frac{\partial f}{\partial C^n} - \epsilon^2\nabla^2 C^n\\
				C^{n+1} &= C^n + D\Delta t\nabla^2\mu
			\end{align*}

		\subsubsection{Convex splitting}
			This advanced example is discussed in Chapter~\ref{ch:advanced}.
			

	\subsection{Model A}
	\subsection{Model B}
	\subsection{Solidification}
	\subsection{Spinodal}


\section{Differential Equations}
	\subsection{Poisson}

\section{Statistical Mechanics}
	\subsection{Heisenberg}
	\subsection{Ising}
	\subsection{Potts}

